% Этот шаблон документа разработан в 2014 году
% Данилом Фёдоровых (danil@fedorovykh.ru) 
% для использования в курсе 
% <<Документы и презентации в LaTeX>>, записанном НИУ ВШЭ
% для Coursera.org: http://coursera.org/course/latex .
% Исходная версия шаблона --- 
% https://www.writelatex.com/coursera/latex/1.2

\documentclass[a4paper,10pt]{article} % добавить leqno в [] для нумерации слева

%%% Работа с русским языком
\usepackage{float}
\usepackage{cmap}					% поиск в PDF
\usepackage{mathtext} 				% русские буквы в формулах
\usepackage[T2A]{fontenc}			% кодировка
\usepackage[utf8]{inputenc}			% кодировка исходного текста
\usepackage[english,russian]{babel}	% локализация и переносы
\usepackage[top=2cm,bottom=2cm,bindingoffset=0cm]{geometry}

%%% Дополнительная работа с математикой
\usepackage{amsmath,amsfonts,amssymb,amsthm,mathtools} % AMS
\usepackage{centernot}
\usepackage{icomma} % "Умная" запятая: $0,2$ --- число, $0, 2$ --- перечисление

%% Номера формул
\mathtoolsset{showonlyrefs=true} % Показывать номера только у тех формул, на которые есть \eqref{} в тексте.

%% Шрифты
\usepackage{euscript}	 % Шрифт Евклид
\usepackage{mathrsfs} % Красивый матшрифт
\usepackage{systeme}

\begin{document}
	\section{Криптография}
	\subsection{}
	Постановка задачи. Простейшие криптосистемы. Сдвиг и афинное преобразование.
	Частотный анализ. Биграммы.
	
	\subsection{}
	
	\subsection{}
	Необходимые сведения из теории чисел. Обратимость вычета по данному модулю.
	Алгоритм нахождения обратного элемента. Малая теорема Ферма. функция Эйлера 
	и теорема Эйлера. Китайская теорема об остатках. Возведение в степень методом
	повторного возведения в квадрат.  
	\\\\
	Вычет a называется обратимым по модулю N, если сущетсвует вычет $x$ такой, 
	что 
	\begin{equation}
		ax \equiv 1 \; (mod \; N) 
	\end{equation}
	Вычет является обратимым тогда и только тогда, когда он взаимно прост с модулем
	($НОД(a, N) = 1$).
	\\\\
	Теорема Ферма утверждает, что если $p$ - простое число и $a$ - целое число, 
	не делящееся на $p$, то
	\begin{equation}
		a^{p-1} \equiv 1 \; (mod \; p);
	\end{equation}
	\\\\
	Функция Эйлера $\varphi(n)$ — мультипликативная арифметическая функция, равная количеству натуральных чисел, меньших $n$ n и взаимно простых с ним. При этом полагают по определению, что число 1 взаимно просто со всеми натуральными числами, и $\varphi(1) = 1$. Пример: $\varphi(24) = 8$: $1, 5, 7, 11, 13, 17, 19, 23$.
	\\\\
	Теорема Эйлера гласит, что если $a$ и $m$ взаимно просты, то $a^{\varphi(m)} 
	\equiv 1 \; (mod \; m)$. Малая теорема Ферма является следствием теореми Эйлера.
	\\\\\
	Китайская теорема об остатках. Пусть $n_{1}, n_{2}, ..., n_{k}$ - некоторые попарно взаимно простые числа, а $r_{1}, r_{2}, ..., r_{k}$ - некоторые целые 
	числа. Тогда существует такое целое число M, что оно будет решением системы 
	уравнений:
	
	\begin{equation}
	\[
	\systeme*{
		M \equiv r_{1} \; (mod \; n_{1}), 
		M \equiv r_{2} \; (mod \; n_{2}),
		\ldots,
		M \equiv r_{k} \; (mod \; n_{k})
	}
	\]
	\end{equation}
	Причём это решение единственно по модулю $n_{1} \cdot n_{2} \cdot ... \cdot n_{k}$
	\\\\
	Метод повторного возведения в квадрат. Дальше идут мои личные объяснения.
	Пусть нам нужно возвести чилсло $a$ в степень $n$. Представим $n$ как 
	сумму степеней двойки. Пример: $51 = 32 + 16 + 2 + 1$. Мы будем вычислять
	$a^{n}$ циклом из $n$ итераций. На итерации $i = \overline{0, n - 1}$ будет вычисляться $a^{2^{i}}$. Причём это будет сделано с помощью уже полученного на предыдущей итерации результата ($a^{2^{i}} =(a^{2^{i-1}})^{2}$). Переменная результата будет инициализирована единицей и будет домножаться на $a^{2^{i}}$ каждый раз, когда $i$ слева бит числа n не равен нулю. Таким образом, мы возведём число в степень $n$ примерно за $\log_{2}{n}$ операций.
	
	\subsection{}
	Группа — множество, на котором определена ассоциативная бинарная операция, причём для этой операции имеется нейтральный элемент (аналог единицы для умножения), и каждый элемент множества имеет обратный.
	\\\\
	Конечная группа - группа с ограниченным числом элементов. Пример конечной группы - вычеты по модулю $n$.
	\\\\
	Пусть $G$ - конечная группа, $m = |G|$ (порядок группы, количество элементов).
	Теорема: $\forall g \in G: g^{m} = e$.
	\\\\
	Порядок элемента $g$ (записывают, как $ord(g)$) - наименьшее натуральное $s$ такое, что $g^{s} = e$. Считают, что $ord(g) = \infty$, если такого $s$ не существует.
	\\\\
	Группа $G$ называется циклической, если $\exists g \in G: G = \{g^{k}, k \in \mathbb{Z}\}$. Пример циклической группы - вычеты по модулю $n$ с операцией сложения.
	
	\subsection{}
	Задача дискретного логарифмирования и система Диффи-Хеллмана обмена ключами.
	\\\\
	Задача дискретного логарфмирования. Пусть $G$ - конечная группа и $g \in G$.
	Задача: для $h \in \{g^{s}, s \in \mathbb{Z}\}$ найти натуральное k такое, что
	$h = g^{k}$. $k$ - дискретный логарифм элемента $h$ по основанию $g$.
	Замечание: такое $k$ - не единственное, так как $g^{m} = e \implies 
	g^{k} = g^{m + k} = g^{2m + k} = ... = g^{nm + k}, n \in \mathbb{N}$.
	Основная фишечка: возведение в степень быстрое, а нахождение логарифма - долгое.
	\\\\
	Система Диффи-Хеллмана (1976). Все знаю конечную группу $G$ и элемент $g$ достаточно большого порядка. $G = (\mathbb{Z}_{p}\backslash\{0\}, \times)$. 
	Важно, что $|G| = p - 1$, $p$ - большое простое. Эта группа циклическая.
	Что происходит? Алиса фиксирует своё некоторое натуральное число $a$, держит его в секрете, но вычисляет и выкладывает значение $g^{a}$. После этого любые 2 участника сформируют у себя общее число $g^{ab} = (g^{a})^{b} = (g^{b})^{a}$.
	На деле тут нет передачи информации. 
	\subsection{}
	\subsection{}
	\subsection{}
	\subsection{}
	\subsection{}
	Проверка числа на простоту и проблема факторизации. Решето Эратосфена.
	Псевдопростые числа и числа Кармайкла. Метод Поклингтона. $(p - 1)$ метод
	Полларда. 
	\\\\
	Псевдопростые числа и числа Кармайкла. Если $p$ - простое, то $\forall a < p:
	a^{p-1} \equiv 1 \; (mod \; p)$ (малая теорема Ферма). Метод Кармайкла позволяет точно сказать, является ли число непростым, но не позволяет утверждать обратного. Если $\exists a < p$ такое, что $a^{p - 1} \centernot\equiv 1 \; (mod \; p)$, то число $p$ - не простое. Если для данного 
	$p$ $\forall a < p: a^{p-1} \equiv 1 \; (mod \; p)$, но при этом само число не являетя простым, то его называют числом Кармайкла (пример: 561).  
	\\\\
	Метод Поклингтона проверки числа на простоту. Предположим, что у числа
	$n - 1$ есть простой делитель $p > \sqrt{n} - 1$. Если $\exists a$ (целое) такое, что выполены 2 условия:
	
	\begin{enumerate}
		\item $a^{n-1} \equiv 1 \; (mod \; n)$
		\item $(a^{\frac{n-1}{p}} - 1, n) = 1$
	\end{enumerate}
	 То число n - простое. 
	 \\\\
	 $(p-1)$-метод Полларда разложения числа на множители. Выберем число m, 
	 которое делится на все натуральные числа $\leq c$ (Пример: $c!$).
	 Возьмём q такое, что $2 \leq q \leq n -2$. Вычислим $q^{m} \; (mod \; n)$.
	 Если $q^{m} \neq 1 \; (mod \; n)$, продолжаем. Вычисляем $d = НОД(q^{m} - 1, n)$. Если $d \neq 1$, то $n = d \cdot \frac{n}{d}$.
	 
	 

\end{document}