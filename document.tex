% Этот шаблон документа разработан в 2014 году
% Данилом Фёдоровых (danil@fedorovykh.ru) 
% для использования в курсе 
% <<Документы и презентации в LaTeX>>, записанном НИУ ВШЭ
% для Coursera.org: http://coursera.org/course/latex .
% Исходная версия шаблона --- 
% https://www.writelatex.com/coursera/latex/1.2

\documentclass[a4paper,10pt]{article} % добавить leqno в [] для нумерации слева

%%% Работа с русским языком
\usepackage{float}
\usepackage{cmap}					% поиск в PDF
\usepackage{mathtext} 				% русские буквы в формулах
\usepackage[T2A]{fontenc}			% кодировка
\usepackage[utf8]{inputenc}			% кодировка исходного текста
\usepackage[english,russian]{babel}	% локализация и переносы
\usepackage[top=2cm,bottom=2cm,bindingoffset=0cm]{geometry}

%%% Дополнительная работа с математикой
\usepackage{amsmath,amsfonts,amssymb,amsthm,mathtools} % AMS
\usepackage{centernot}
\usepackage{icomma} % "Умная" запятая: $0,2$ --- число, $0, 2$ --- перечисление

%% Номера формул
\mathtoolsset{showonlyrefs=true} % Показывать номера только у тех формул, на которые есть \eqref{} в тексте.

%% Шрифты
\usepackage{euscript}	 % Шрифт Евклид
\usepackage{mathrsfs} % Красивый матшрифт
\usepackage{systeme}

\begin{document}
	\section{Криптография}
	\subsection{}
	Постановка задачи. Простейшие криптосистемы. Сдвиг и афинное преобразование.
	Частотный анализ. Биграммы.
	
	\subsection{}
	
	\subsection{}
	
	Вычет a называется обратимым по модулю N, если сущетсвует вычет $x$ такой, 
	что 
	\begin{equation}
		ax \equiv 1 \; (mod \; N) 
	\end{equation}
	Вычет является обратимым тогда и только тогда, когда он взаимно прост с модулем
	($НОД(a, N) = 1$).
	\\\\
	Теорема Ферма утверждает, что если $p$ - простое число и $a$ - целое число, 
	не делящееся на $p$, то
	\begin{equation}
		a^{p-1} \equiv 1 \; (mod \; p);
	\end{equation}
	
	Функция Эйлера $\varphi(n)$ — мультипликативная арифметическая функция, равная количеству натуральных чисел, меньших $n$ n и взаимно простых с ним. При этом полагают по определению, что число 1 взаимно просто со всеми натуральными числами, и $\varphi(1) = 1$. Пример: $\varphi(24) = 8$: $1, 5, 7, 11, 13, 17, 19, 23$.
	
	Теорема Эйлера гласит, что если $a$ и $m$ взаимно просты, то $a^{\varphi(m)} 
	\equiv 1 \; (mod \; m)$. Малая теорема Ферма является следствием теореми Эйлера.
	
	Китайская теорема об остатках. Пусть $n_{1}, n_{2}, ..., n_{k}$ - некоторые попарно взаимно простые числа, а $r_{1}, r_{2}, ..., r_{k}$ - некоторые целые 
	числа. Тогда существует такое целое число M, что оно будет решением системы 
	уравнений:
	
	\begin{equation}
	\systeme*{
		M \equiv r_{1} \; (mod \; n_{1}), 
		M \equiv r_{2} \; (mod \; n_{2}),
		.,
		M \equiv r_{k} \; (mod \; n_{k})}
	\end{equation}
	Причём это решение единственно по модулю $n_{1} \cdot n_{2} \cdot ... \cdot n_{k}}

\end{document}